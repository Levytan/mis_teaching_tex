% !TEX program = xelatex
\documentclass[a4paper]{article}
\usepackage[margin=3cm]{geometry}
\usepackage{amssymb}
\usepackage{array}
\usepackage{booktabs}
\usepackage[table]{xcolor}
\usepackage{unicode-math}
\usepackage{amsmath}
\usepackage[linesnumbered,ruled]{algorithm2e}
\usepackage{hyperref}
\usepackage{geometry}
\usepackage{indentfirst}
\usepackage{multirow}
\usepackage{enumitem}
\usepackage{graphicx}
\usepackage{fancyhdr}

\usepackage{fontspec}

\setmainfont{Merriweather}
\setmathfont{STIX Two Math}[Scale=MatchLowercase]
\setmonofont{Consolas}[Scale=MatchLowercase]
\setlist[itemize]{noitemsep, topsep=0.5em}
\geometry{
	left = 25mm,
	right = 15mm,
	top = 20mm,
    bottom = 20mm,
    includeheadfoot=true
}
\fancypagestyle{style}{
    \fancyhf{}
    \fancyfoot[C]{\textit{\ssp Cán bộ coi thi không giải thích gì thêm}}
    \renewcommand{\headrulewidth}{0pt}  
    \renewcommand{\footrulewidth}{0pt}}
\pagestyle{style}
\newfontfamily\ssp{Source Serif Pro}

\begin{document}
{\ssp \begin{center}
    \begin{tabular}{p{0.25\linewidth} p{0.6\linewidth}}
        \begin{center}
            \includegraphics[height = 1.2cm]{hutech.png} \newline
            \textbf{KHOA HỆ THỐNG THÔNG TIN QUẢN LÝ}
        \end{center}
        &
        \begin{center}
            \begin{tabular}[t]{l l}
                \textbf{ĐỀ THI:} &\textbf{HỌC KỲ 2B LẦN 1 NĂM HỌC 2020-2021} \\
                Ngành/Lớp: &HỆ THỐNG THÔNG TIN QUẢN LÝ \\
                Tên học phần: &TƯ DUY TÍNH TOÁN \\
                Mã học phần: &MAT208 \hspace{2em} Số tín chỉ: 03 \\
                Thời gian làm bài: &90 phút \\
                Mã đề (nếu có): & 01 \\
                \multicolumn{2}{l}{\textbf{SỬ DỤNG TÀI LIỆU:\hspace{1em} CÓ ✓\hspace{1em} KHÔNG ☐}}\\
            \end{tabular}
        \end{center}
        \\
        \hline
    \end{tabular}
\end{center}
}
Có 3 cách thường dùng để biểu diễn thuật toán, đó là: dùng ngôn ngữ tự nhiên, dùng lưu đồ và dùng mã giả.
    Trong đó, cách dùng mã giả là phổ biến nhất.

    Mã giả là cách biểu diễn thuật toán bằng cách kết hợp ngôn ngữ tự nhiên và các cấu trúc lập trình (cấu trúc rẽ nhánh, cấu trúc lặp).
    Mã giả không có tiêu chuẩn chung, nhưng khi viết phải thể hiện được đầy đủ, rõ ràng thuật toán.

    Khi viết mã giả, người ta có thể mượn từ khóa của một ngôn ngữ lập trình nào đó để thể hiện cấu trúc lập trình.
    Trong tài liệu này, các thuật toán sẽ được biểu diễn dưới dạng mã giả theo phong cách python như ví dụ dưới đây.
\end{document}