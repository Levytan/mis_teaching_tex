% !TEX program = xelatex
\documentclass[11pt]{beamer}

\usepackage{unicode-math}
\usepackage{amsmath}
\usepackage{amsfonts}
\usepackage{amssymb}
\usepackage[style=ddmmyyyy]{datetime2}
\usepackage{hyperref}
\usepackage{fontspec}

\makeatletter
\def\input@path{{../../theme/}}
\makeatother

\usetheme{mis}

\setsansfont{Rosario}[Numbers=OldStyle]
\setmathfont{STIX Two Math}[Scale=MatchLowercase]
\setmonofont{Consolas}[Scale=MatchLowercase]

\author{Lê Thành Văn}
\title{Lập trình nâng cao}
\institute{Khoa Hệ thống thông tin quản lý}
\date{\today}
% package setting
\hypersetup {
	colorlinks = true
}
%\usecolortheme{seahorse}
% short hand
\newcommand{\pandas}{\texttt{pandas}}
%
\AtBeginSection{
  \frame{
    \sectionpage
  }
}
%
\begin{document}

\begin{frame}
\titlepage
\end{frame}

\section{Giới thiệu}
\subsection{Học phần}
\begin{frame}{Học phần}
Học phần này bao gồm 2 phần chính :
\begin{itemize}
\item Làm việc với \texttt{DataFrame} (thư viện \pandas\ của Python).
\item Giới thiệu về machine learning và các thuật toán cơ bản.
\end{itemize}
\end{frame}
\subsection{Mục đích}
\begin{frame}{Mục đích}
\begin{itemize}
\item Nắm bắt được các giai đoạn và kỹ năng cơ bản khi phân tích dữ liệu
\item Nắm bắt các thuật toán machine learning cơ bản
\end{itemize}
\end{frame}
\section{Thư viện \pandas}
\subsection{Giới thiệu về \pandas}
\begin{frame}{Giới thiệu về \pandas}
\begin{itemize}
\item Là một thư viện trên Python được phát triển bởi Wes McKinney trong năm 2008.
\item Là thư viện chuẩn cho việc phân tích dữ liệu khi dùng Python.
\end{itemize}
\end{frame}
\subsection{Tính năng của \pandas}
\begin{frame}{Tính năng của \pandas}
\begin{itemize}
\item Có thể xử lý tập dữ liệu khác nhau về định dạng.
\item Có khả năng đọc dữ liệu từ nhiều nguồn khác nhau (csv, db/sql, excel, ...).
\item Có thể xử lý nhiều phép toán cho tập dữ liệu
\item Xử lý mất mát dữ liệu
\end{itemize}
\end{frame}
\subsection{Các kiểu dữ liệu trong \pandas}
\begin{frame}{Các kiểu dữ liệu trong \pandas}
\begin{itemize}
\item \texttt{Series} : kiểu dữ liệu dạng mảng 1 chiều
\item \texttt{DataFrame} : kiểu dữ liệu mảng 2 chiều
\item \texttt{Panel} : kiểu dữ liệu mảng 3 chiều
\end{itemize}
\end{frame}
\subsection{Cài đặt \pandas}
\begin{frame}{Cài đặt \pandas}
\begin{itemize}
\item Thường sẽ được đi kèm khi cài đặt anaconda.
\item Nếu không sử dụng anaconda thì chạy \texttt{pip install pandas} trong command prompt (cmd)
\end{itemize}
\end{frame}
\subsection{Sử dụng \pandas\ trong python}
\begin{frame}{Sử dụng \pandas\ trong python}
Để sử dụng \pandas\, thêm dòng \texttt{import pandas as pd}\ ở đầu file
\end{frame}
\section{Machine Learning}
\subsection{Giới thiệu}
\begin{frame}{Giới thiệu}
Nói một cách đơn giản, machine learning (máy học) là những phương pháp cung cấp cho máy tính cách học hỏi mà không cần hướng dẫn chi tiết từng bước.\\
Machine learning là một ĩnh vực của Artificial Intelligence (AI, trí tuệ nhân tạo)
\end{frame}
\subsection{Phương pháp}
\begin{frame}{Phương pháp}
Machine learning bao gồm các phương pháp sau (theo cách học) :
\begin{itemize}
\item Supervised Learning : học có giám sát (phân loại (classification), hồi quy (regression), ...)
\item Unsupervised Learning : học không giám sát (phân nhóm (clustering), ...)
\item Reinforcement Learning : học củng cố (học sâu (deep learing), mạng nơ-ron (neural networks), ...)
\end{itemize}
\end{frame}
\subsection{Các vấn đề}
\begin{frame}
Hiện tại, có rất nhiều vấn đề được giải quyết bằng machine learning
\begin{itemize}
\item Xử lý ảnh (gắn thẻ hình ảnh, nhận diện ký tự, nhận diện khuôn mặt, ...)
\item Xử lý văn bản (phát hiện spam, phân tích ngữ nghĩa, ...)
\item Khai phá dữ liệu (gom nhóm, dự đoán, ...)
\end{itemize}
\end{frame}
\subsection{Quy trình cơ bản}
\begin{frame}{Quy trình cơ bản}
  \begin{enumerate}
    \item Thu thập dữ liệu
    \item Làm sạch dữ liệu
    \item Khám phá dữ liệu
    \item Chuyển đổi dữ liệu
    \item Xây dựng mô hình
    \item Đánh giá và điều chỉnh
  \end{enumerate}

\end{frame}
\subsection{Machine learning trong Python}
\begin{frame}{Machine learning trong Python}
Trong python có nhiều package về machine learning như \texttt{scikit-learn}, \texttt{TensorFlow}, ...
\end{frame}
\section{Tài liệu tham khảo}
\subsection{\pandas}
\begin{frame}{\pandas}
\begin{itemize}
\item \href{https://www.tutorialspoint.com/python_pandas/index.htm}{Hướng dẫn về \pandas}
\item \href{https://pandas.pydata.org/pandas-docs/stable/}{\pandas ' documentation} : tài liệu của \pandas{} (phương thức, kiểu dữ liệu, ...).
\item \href{https://stackoverflow.com/questions/tagged/pandas}{Stack Overflow về \pandas{}} : là nơi sẽ trả lời các khó khăn khi làm việc với \pandas
\end{itemize}
\end{frame}
\subsection{Machine Learning}
\begin{frame}{Machine Learning}
\begin{itemize}
\item \href{https://scikit-learn.org/stable/documentation.html}{scikit-learn's documentation} : tài liệu của scikit-learn (thư viện machine learning của microsoft)
\item \href{https://www.tensorflow.org/tutorials/}{Hướng dẫn về TensorFlow} : thư viện machine learning của Google
\end{itemize}
\end{frame}
\subsection{Khóa học}
\begin{frame}{Khóa học}
\begin{itemize}
\item \href{https://courses.edx.org/courses/course-v1:Microsoft+DAT210x+6T2016/course/}{Programming with Python for Data Science} : khóa học về \pandas{} và một số thuật toán cơ bản trong machine learning.
\item \href{https://www.coursera.org/learn/machine-learning}{Khóa học machine learning của Andrew Ng}
\end{itemize}
\end{frame}
\subsection{Khác}
\begin{frame}{Khác}
Sách :
\begin{itemize}
\item \href{https://pdfs.semanticscholar.org/ce61/5ae61d67db8537e981a0a08da7f0f2ff1cee.pdf?_ga=2.176850727.342533066.1573446520-1563216005.1573446520}{Understanding Machine Learning: From Theory to Algorithms}
\end{itemize}
Internet
\begin{itemize}
\item \href{https://machinelearningcoban.com/}{Machine Learning cơ bản}
\end{itemize}
\end{frame}
\end{document} 
