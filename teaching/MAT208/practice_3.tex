\documentclass[12pt, a4paper]{article}

\usepackage{hyperref}
\usepackage{array}
\usepackage[style=ddmmyyyy]{datetime2}
\usepackage{minted}
\usepackage{geometry}
\usepackage{fontspec}
\usepackage{setspace}
\usepackage{unicode-math}
\usepackage{xcolor}
\usepackage{indentfirst}
\usepackage{enumitem}
\usepackage{graphicx}

%fontspec settings
\setmainfont{Source Serif Pro}
\setmathfont{STIX Two Math}[Scale=MatchLowercase]
\setmonofont{Consolas}[Scale=MatchLowercase]
\setlist[itemize]{noitemsep, topsep=0.5em}
%geometry
\geometry{
	left = 20mm,
	right = 15mm,
	top = 20mm,
	bottom=20mm,
}

\title{Bài tập thực hành 3}
\date{2021-06-29}
\author{}
\begin{document}
    \onehalfspacing
    \maketitle
    \textbf{Bài 1}. Cho chương trình python dùng để chia danh sách làm ba phần như sau:
    \begin{minted}{python}
        def chia_mang(A):
            """
            Phần tử chốt được chọn là phần tử đứng cuối danh sách.
            Input: Danh sách cần chia.
            Output: Gồm 3 danh sách:
                - left: gồm những phần tử nhỏ hơn chốt.
                - right: gồm những phần tử lớn hơn chốt.
                - [pivot]: danh sách chứa một phần tử chốt duy nhất.
            """
            left = right = []
            pivot = A[-1]
            for e in A[:-1]: # lặp từng phần tử trong A trừ phần tử cuối cùng
                if e < pivot:
                    left = left + [e]
                else:
                    right = right + [e]
            return left, [pivot], right
    \end{minted}
    Viết lại chương trình sắp xếp nhanh sử dụng chương trình chia mảng ở trên.

    \textbf{Bài 2}. Trong một danh sách số, ta định nghĩa một phần tử là cực đại nếu như 
    nó lớn hơn phần tử liền trước và liền sau của chính phần tử đó.
    Viết chương trình in ra các phần tử cực đại trong một danh sách số cho trước.

    \textit{Ví dụ}. 
    \begin{itemize}
        \item $A = [8, 9, 10, 6, 5, 3]$, phần tử cực đại sẽ là $10$.
        \item $A = [8, 7, 10, 2, 6, 5]$, phần tử cực đại sẽ là $8$ và $10$.
    \end{itemize}
    
    \textbf{Bài 3}. Cho danh sách số $A$, viết chương trình tìm chỉ số $k$ sao cho $A[k+1]-A[k]$ là nhỏ nhất.
    \newpage
    \textbf{Bài 4}. Cho thuật toán tìm kiếm số $k$ trong danh sách số $A$ như sau:
    \begin{enumerate}
        \item Đặt $b = 1$.
        \item \textbf{Gấp đôi} $b$ cho đến khi $k < A[b]$ hoặc $b > len(A)$.
        \item Tìm kiếm nhị phân với điểm bắt đầu là $b//2$ và điểm kết thúc là $min(b, len(A))$.
    \end{enumerate}
    Viết chương trình python cho thuật toán trên.
\end{document}