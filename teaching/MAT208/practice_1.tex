\documentclass[12pt, a4paper]{article}

\usepackage{hyperref}
\usepackage{array}
\usepackage[style=ddmmyyyy]{datetime2}
\usepackage{minted}
\usepackage{geometry}
\usepackage{fontspec}
\usepackage{setspace}
\usepackage{unicode-math}
\usepackage{xcolor}
\usepackage{indentfirst}
\usepackage{enumitem}
\usepackage{graphicx}

%fontspec settings
\setmainfont{Source Serif Pro}
\setmathfont{STIX Two Math}[Scale=MatchLowercase]
\setmonofont{Consolas}[Scale=MatchLowercase]
\setlist[itemize]{noitemsep, topsep=0.5em}
%geometry
\geometry{
	left = 20mm,
	right = 15mm,
	top = 20mm,
	bottom=20mm,
}

\title{Bài tập thực hành 1}
\date{\today}
\begin{document}
    \onehalfspacing
    \maketitle
    Viết các chương trình sau bằng đệ quy.

    \textbf{Bài 1}. Tính tổng các số trong một danh sách số.

    \textbf{Bài 2}. Tính ước chung lớn nhất của hai số $a$ và $b$. 
    Giả sử $a > b$, nếu $a$ chia hết cho $b$ thì $\text{UCLN}(a, b) = b$.
    Nếu không thì $\text{UCLN}(a, b) = \text{UCLN}(b, a \% b)$.
    
    \textbf{Bài 3}. Đếm số lần xuất hiện của một ký tự trong một chuỗi bất kỳ.
    Ví dụ:
    \begin{itemize}
        \item \texttt{\detokenize{dem_ky_tu('abcabacaa', 'a') -> 5}}.
        \item \texttt{\detokenize{dem_ky_tu('abcabacaa', 'd') -> 0}}.
    \end{itemize}

    \textbf{Bài 4}. In ra màn hình các chuỗi nhị phân có độ dài 5 mà không có 2 ký tự 1 đứng liền nhau.
    \newpage
    \textbf{Bài 5}. Cho chương trình tìm kiếm nhị phân trong python như sau:
    \begin{minted}{python}
        def tim_nhi_phan(A, k, l, r):
            """
            A: danh sách (list) số
            k: số cần tìm
            l: vị trí bắt đầu
            r: vị trí kết thúc
            Gọi tim_nhi_phan(A, k, 0, len(A) - 1) để tìm kiếm trong toàn danh sách
            """
            # tìm điểm chia đôi
            m = (l + r) // 2
            # giải quyết từng trường hợp
            if l > r: # không tìm thấy
                return -1
            if A[m] == k:
                return m
            elif A[m] > k: # giá trị cần tìm nằm bên trái điểm chia
                return tim_nhi_phan(A, k, l, m - 1)
            else: # giá trị cần tìm nằm bên phải điểm chia
                return tim_nhi_phan(A, k, m + 1, r)
    \end{minted}
    Hãy thay đổi chương trình trên để thực hiện các thuật toán:
    \begin{itemize}
        \item Tìm kiếm tam phân: tương tự như tìm kiếm nhị phân, 
        nhưng thay vì chia đoạn cần tìm làm hai phần, chia nó thành ba phần.
        \item Tìm kiếm nội suy: tương tự như tìm kiếm nhị phân, 
        nhưng điểm chia sẽ phụ thuộc vào giá trị của số cần tìm $k$.
        Trong thuật toán này, chúng ta sẽ dùng điểm chia là \textit{số lớn nhất nhỏ hơn}
        $\dfrac{k - A_l}{A_r - A_l} \times (r - l)$
    \end{itemize}
\end{document}