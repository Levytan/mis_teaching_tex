\documentclass[12pt, a4paper]{article}

\usepackage{hyperref}
\usepackage{array}
\usepackage[style=ddmmyyyy]{datetime2}
\usepackage{geometry}
\usepackage{fontspec}
\usepackage{setspace}
\usepackage{unicode-math}
\usepackage{xcolor}
\usepackage{indentfirst}
\usepackage{enumitem}
\usepackage{graphicx}

%fontspec settings
\setmainfont{Source Serif Pro}
\setmathfont{STIX Two Math}[Scale=MatchLowercase]
\setmonofont{Consolas}[Scale=MatchLowercase]
\setlist[itemize]{noitemsep, topsep=0.5em}
%geometry
\geometry{
	left = 20mm,
	right = 15mm,
	top = 20mm,
	bottom=20mm,
}

\title{Kiểm tra bổ sung}
\date{\today}
\author{}
\begin{document}
    \onehalfspacing
    \maketitle
    \textbf{Bài 1}. Cho số tự nhiên $n$, viết chương trình tìm số tự nhiên $k$ sao cho $k^2 \leq n < (k+1)^2$.

    \textbf{Bài 2}. Cho hai danh sách $A$ và $B$ được sắp xếp tăng dần,
    viết chương trình tạo ra một danh sách $C$ cũng được sắp xếp tăng dần với tính chất 
    $C[0]$, $C[2]$, $C[4]$, ... thuộc danh sách $A$ và $C[1]$, $C[3]$, $C[5]$, ... thuộc danh sách $B$.

    \textit{Ví dụ}: Cho $A = [10, 15, 25]$, $B = [1, 5, 20, 30]$, ta có lấy ra $C = [10, 20, 15, 30]$. 
\end{document}