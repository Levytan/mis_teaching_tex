\documentclass[12pt, a4paper]{article}

\usepackage{hyperref}
\usepackage{array}
\usepackage[style=ddmmyyyy]{datetime2}
\usepackage{minted}
\usepackage{geometry}
\usepackage{fontspec}
\usepackage{setspace}
\usepackage{unicode-math}
\usepackage{xcolor}
\usepackage{indentfirst}
\usepackage{enumitem}
\usepackage{graphicx}

%fontspec settings
\setmainfont{Source Serif Pro}
\setmathfont{STIX Two Math}[Scale=MatchLowercase]
\setmonofont{Consolas}[Scale=MatchLowercase]
\setlist[itemize]{noitemsep, topsep=0.5em}
%geometry
\geometry{
	left = 20mm,
	right = 15mm,
	top = 20mm,
	bottom=20mm,
}
\title{Bài tập thực hành 4}
\date{2021-08-17}
\author{}
\begin{document}
    \onehalfspacing
    \maketitle
    \textbf{Bài 1}. Cho danh sách số $A$, ta sẽ tạo một danh sách số mới từ $A$
    bằng cách lấy hiệu hai phần tử liên tiếp trong $A$. 
    Ví dụ $A = [1, 4, 9, 16]$ thì danh sách mới sẽ là $[3, 5, 7]$.
    Tiếp tục thực hiện việc tạo danh sách này cho đến khi nhận được danh sách chỉ chứa các phần tử giống nhau.
    Viết chương trình python để đếm số lần thực hiện với một danh sách số cho trước.
    
    \textit{Ví dụ}:
    \begin{itemize}
        \item $A = [1, 4, 9, 16] \rightarrow [3, 5, 7] \rightarrow [2, 2]$, chương trình trả về 2.
        \item $A = [1, 8, 27, 64] \rightarrow [7, 19, 37] \rightarrow [12, 18] \rightarrow [6]$, chương trình trả về 3.
    \end{itemize}

    \textit{Gợi ý}: Dùng hàm \texttt{set()} của python.

    \textbf{Bài 2}. Một cửa hàng bán nước giải khát với giá $c$ đồng một chai.
    Cửa hàng đó cũng cho đổi vỏ chai đã dùng lấy một chai nước mới với tỷ lệ $e$ vỏ chai đổi một chai mới.
    Trong túi của bạn có $m$ đồng. Viết chương trình python từ ba số $c$, $e$ và $m$ in ra số lượng chai tối đa mà một người có thể mua.

    \textit{Ví dụ 1}: Cho $c = 2$, $e = 2$ và $m = 16$, chương trình sẽ in ra 15. 
    Với 16 đồng thì mua được 8 chai, 8 vỏ này đổi được 4 chai mới, 4 vỏ lại đổi được 2 chai,
    2 vỏ lại đổi được 1 chai. Tổng cộng là 8 + 4 + 2 + 1 = 15 chai.

    \textit{Ví dụ 2}: Cho $c = 1$, $e = 3$ và $m = 15$, chương trình sẽ in ra 22.
    Với 15 đồng sẽ mua được 15 chai, 15 vỏ đổi được 5 chai mới, 5 vỏ đổi được 1 chai mới (dư 2 vỏ chai),
    chai mới đổi và 2 vỏ chai còn dư là 3 vỏ nên đổi được 1 chai nữa. Tổng cộng là 15 + 5 + 1 + 1 = 22 chai.
\end{document}