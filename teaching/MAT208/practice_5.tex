\documentclass[12pt, a4paper]{article}

\usepackage{hyperref}
\usepackage{array}
\usepackage[style=ddmmyyyy]{datetime2}
\usepackage{minted}
\usepackage{geometry}
\usepackage{fontspec}
\usepackage{setspace}
\usepackage{unicode-math}
\usepackage{xcolor}
\usepackage{indentfirst}
\usepackage{enumitem}
\usepackage{graphicx}

%fontspec settings
\setmainfont{Source Serif Pro}
\setmathfont{STIX Two Math}[Scale=MatchLowercase]
\setmonofont{Consolas}[Scale=MatchLowercase]
\setlist[itemize]{noitemsep, topsep=0.5em}
%geometry
\geometry{
	left = 20mm,
	right = 15mm,
	top = 20mm,
	bottom=20mm,
}
\title{Bài tập thực hành 5}
\date{2021-07-14}
\author{}
\begin{document}
    \onehalfspacing
    \maketitle
	\textbf{Bài 1}. Cho một chuỗi gồm các ký tự từ `a' đến `z' (in thường), 
	ta có thuật toán sắp xếp lại chuỗi này thành chuỗi mới với các ký tự xuất hiện theo thứ tự bảng chữ cái
	(tức là từ chuỗi `asdfg' sẽ thành `adfgs') như sau.
	\begin{enumerate}
		\item Tạo một danh sách $C$ gồm 26 số 0 (\texttt{C = [0] * 26}). 
		\item Cập nhật $C$ là số lần xuất hiện từng chữ cái trong chuỗi đã cho, với quy ước, 
		$C[0]$ là số lần xuất hiện của ký tự `a',
		$C[1]$ là số lần xuất hiện của ký tự `b',
		..., $C[25]$ là số lần xuất hiện của ký tự `z'.
		\item Tiếp tục cập nhật $C$ theo nguyên tắc $C[i] = C[i] + C[i-1]$ với $i = 1, 2, ..., 25$.
		\item Tạo mỗi chuỗi mới $s\_new$ gồm toàn \textit{các khoảng trắng} với độ dài bằng độ dài chuỗi đã cho.
		\item Với \textbf{từng ký tự} $ch$ trong chuỗi đã cho, lấy ra $C[i]$ tương ứng với ký tự đó 
		(tức là nếu là `a' thì lấy $C[0]$, `b' thì lấy $C[1]$, ...), gán $s\_new[C[i]]$ bằng ký tự $ch$ đó
		rồi giảm $C[i]$ đi 1.
	\end{enumerate}
	Viết chương trình python tương ứng cho thuật toán trên.

	\textbf{Bài 2}. Viết chương trình tính $\binom{n}{k}$ ($0 \leq k \leq n$), biết rằng
	$$\binom{n}{k} = \binom{n-1}{k-1} + \binom{n-1}{k},$$
	$$\binom{n}{n} = \binom{n}{0} = 1$$
\end{document}