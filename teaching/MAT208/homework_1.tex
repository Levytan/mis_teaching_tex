\documentclass[12pt, a4paper]{article}

\usepackage{hyperref}
\usepackage{array}
\usepackage[style=ddmmyyyy]{datetime2}
\usepackage{geometry}
\usepackage{fontspec}
\usepackage{setspace}
\usepackage{unicode-math}
\usepackage{xcolor}
\usepackage{indentfirst}
\usepackage{enumitem}
\usepackage{graphicx}

%fontspec settings
\setmainfont{Source Serif Pro}
\setmathfont{STIX Two Math}[Scale=MatchLowercase]
\setmonofont{Consolas}[Scale=MatchLowercase]
\setlist[itemize]{noitemsep, topsep=0.5em}
%geometry
\geometry{
	left = 20mm,
	right = 15mm,
	top = 20mm,
	bottom=20mm,
}

\title{Bài tập về nhà 1 \\ \Large Đáp án Phần Lý thuyết}
\date{}
\begin{document}
    \onehalfspacing
    \maketitle
    \textbf{Bài 1}: Cho hàm số $A(m, n)$ ($m, n$ là các số tự nhiên) được định nghĩa như sau:
    \begin{subequations}
        \begin{align}
            A(0, n) &= n + 1 \label{eqn:line-1}\\
            A(m, 0) &= A(m - 1, 1) \label{eqn:line-2}\\
            A(m, n) &= A(m - 1, A(m, n - 1)) \label{eqn:line-3}
        \end{align}
        \label{eqn:all-lines}
    \end{subequations}
    Tính $A(1, 2)$ và $A(2, 1)$.

    Ta có:
    \begin{align*}
        A(1, 2) &= A(0, A(1, 1)) &&\text{do}\ (\ref{eqn:line-3})\\
        &= A(0, A(0, A(1, 0))) &&\text{do}\ (\ref{eqn:line-3})\\
        &= A(0, A(0, A(0, 1))) &&\text{do}\ (\ref{eqn:line-2})\\
        &= A(0, A(0, 2)) &&\text{do}\ (\ref{eqn:line-1})\\
        &= A(0, 3) &&\text{do}\ (\ref{eqn:line-1})\\
        &= 4 &&\text{do}\ (\ref{eqn:line-1})\\
    \end{align*}

    \begin{align*}
        A(2, 1) &= A(1, A(1, 1)) &&\text{do}\ (\ref{eqn:line-3})\\
        &= A(1, A(0, A(1, 0))) &&\text{do}\ (\ref{eqn:line-3})\\
        &= A(1, A(0, A(0, 1))) &&\text{do}\ (\ref{eqn:line-2})\\
        &= A(1, A(0, 2)) &&\text{do}\ (\ref{eqn:line-1})\\
        &= A(1, 3) &&\text{do}\ (\ref{eqn:line-1})\\
        &= A(0, A(1, 2)) &&\text{do}\ (\ref{eqn:line-3})\\
        &= A(0, 4) &&(A(1, 2)\ \text{tính như trên})\\
        &= 5 &&\text{do}\ (\ref{eqn:line-1})\\
    \end{align*}
    \newpage
    \textbf{Bài 2}: Ta định nghĩa một chuỗi nhị phân là một chuỗi chỉ bao gồm các ký tự `0' và `1'. 
    Gọi $S_n$ là số lượng chuỗi nhị phân có độ dài $n$ mà không có 2 ký tự `1' đứng kế nhau.
    Chứng minh rằng:
    $$S_{n + 2} = S_{n + 1} + S_{n},\quad n \geq 1$$

    \textit{Đáp án}. Để thuận tiện, ta gọi chuỗi nhị phân không có hai ký tự `1' đứng liền nhau là \textit{chuỗi hợp lệ}.
    Theo định nghĩa chuỗi nhị phân, thì một chuỗi nhị phân có độ dài $n + 2$ chỉ có thể bắt đầu bằng `0' hoặc `1'.
    
    Xét chuỗi hợp lệ độ dài $n + 2$ và bắt đầu bằng `0'. 
    Có thể thấy rằng, nếu bỏ ký tự `0' ở đầu này, ta được một chuỗi hợp lệ độ dài $n + 1$.

    Xét chuỗi hợp lệ độ dài $n + 2$ và bắt đầu bằng `1'.
    Do tính chất của chuỗi hợp lệ, ký tự kế tiếp phải là `0'.
    Và nếu ta bỏ đi `10' ở đầu chuỗi, ta được một chuỗi hợp lệ độ dài $n$.

    Vậy, ta có thể kết luận:
    $$S_{n + 2} = S_{n + 1} + S_{n},\quad n \geq 1$$
\end{document}
