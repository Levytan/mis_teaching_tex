\documentclass[12pt, a4paper]{article}

\usepackage{hyperref}
\usepackage{array}
\usepackage[style=ddmmyyyy]{datetime2}
\usepackage{minted}
\usepackage{geometry}
\usepackage{fontspec}
\usepackage{setspace}
\usepackage{unicode-math}
\usepackage{xcolor}
\usepackage{indentfirst}
\usepackage{enumitem}
\usepackage{graphicx}
\usepackage{booktabs}
\usepackage{array}
\usepackage{diagbox}

%fontspec settings
\setmainfont{Source Serif Pro}
\setmathfont{STIX Two Math}[Scale=MatchLowercase]
\setmonofont{Consolas}[Scale=MatchLowercase]
\setlist[itemize]{noitemsep, topsep=0.5em}
%geometry
\geometry{
	left = 20mm,
	right = 15mm,
	top = 20mm,
	bottom=20mm,
}
%
\renewcommand{\arraystretch}{1.2}
\newcolumntype{M}{>{\centering\arraybackslash}m{0.8cm}}
\title{Bài tập thực hành 6}
\date{}
\author{}
\begin{document}
    \onehalfspacing
    \maketitle
    \textbf{Bài 1}. Người ta muốn lát một căn phòng kích thước $m \times n$ ($m$ và $n$ là những số nguyên) 
    bằng những viên gạch vuông với cạnh là lũy thừa của 2 ($1 \times 1$, $2 \times 2$, $4 \times 4$, $8 \times 8$, ...),
	viết chương trình nhập vào 2 số $m$ và $n$ rồi in ra số viên gạch cần dùng.

	\textit{Gợi ý}:
	\begin{itemize}
		\item Xét trường hợp $m = 2x$ và $n = 2y$, ta thấy rằng nếu gấp đôi cạnh của những viên gạch 
		trong cách lát sàn $x \times y$ thì sẽ được cách lát của sàn $2x \times 2y$.
		\item Xét trường hợp $m = 2x+1$ và $n = 2y$, có thể chia căn phòng thành hai phần gồm 
        một phần $1 \times 2y$ và một phần $2x \times 2y$.
        \item Xét trường hợp $m = 2x$ và $n = 2y+1$, có thể chia căn phòng thành hai phần gồm 
        một phần $2x \times 1$ và một phần $2x \times 2y$.
        \item Xét trường hợp $m = 2x+1$ và $n = 2y+1$, có thể chia căn phòng thành bốn phần gồm 
        một phần $2x \times 1$, một phần $1 \times 2y$, một phần $1 \times 1$ và một phần $2x \times 2y$.
	\end{itemize}

    \textit{Kết quả tham khảo}
    \begin{table}[h]
        \centering
        \begin{tabular}{|c|M|M|M|M|M|M|M|M|}
            \hline
            \diagbox{$m$}{$n$} &$2$ &$3$ &$4$ &$5$ &$10$ &$15$ &$20$ &$25$          \\ \hline
            $2$ &1 & & & & & & &     \\ \hline
            $3$ &3 &6 & & & & & &       \\ \hline
            $4$ &2 &6 &1 & & & & &       \\ \hline
            $5$ &4 &9 &5 &10 & & & &       \\ \hline
            $10$ &5 &15 &4 &14 &10 & & &      \\ \hline
            $15$ &9 &24 &9 &24 &24 &48 & &      \\ \hline
            $20$ &10 &30 &5 &25 &14 &39 &10 &      \\ \hline
            $25$ &14 &39 &10 &35 &25 &60 &29 &55      \\ \hline
        \end{tabular}
    \end{table}

    \textbf{Bài 2}. Viết chương trình nhập vào số tự nhiên $n$, in ra màn hình số cách có thể biểu diễn $n$
    dưới dạng tổng của ba số tự nhiên.
    
    \textit{Kết quả tham khảo}
    \begin{table}[h]
        \centering
        \begin{tabular}{@{}lrrrrrrrrrrr@{}}
        \toprule
        $n$ &1 &2 &3 &4 &5 &6 &7 &8 &9 &10  \\ \midrule
        Kết quả &3 &6 &10 &15 &21 &28 &36 &45 &55 &66 \\ \bottomrule
        \end{tabular}
    \end{table}
\end{document}