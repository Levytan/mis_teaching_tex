\documentclass[12pt, a4paper]{article}

\usepackage{hyperref}
\usepackage{array}
\usepackage[style=ddmmyyyy]{datetime2}
\usepackage{minted}
\usepackage{geometry}
\usepackage{fontspec}
\usepackage{setspace}
\usepackage{unicode-math}
\usepackage{xcolor}
\usepackage{indentfirst}
\usepackage{enumitem}
\usepackage{graphicx}

%fontspec settings
\setmainfont{Source Serif Pro}
\setmathfont{STIX Two Math}[Scale=MatchLowercase]
\setmonofont{Consolas}[Scale=MatchLowercase]
\setlist[itemize]{noitemsep, topsep=0.5em}
%geometry
\geometry{
	left = 20mm,
	right = 15mm,
	top = 20mm,
	bottom=20mm,
}
\title{Bài tập thực hành 4}
\date{2021-07-06}
\author{}
\begin{document}
    \onehalfspacing
    \maketitle
    \textbf{Bài 1}. Viết chương trình dùng sắp xếp theo cơ số (radix sort) để sắp xếp một danh sách các chuỗi ký tự.
    Để đơn giản, chuỗi ký tự chỉ bao gồm các ký tự \textbf{in thường} từ `a' đến `z'.

    \textbf{Bài 2}. Tương tự như heap cực đại, ta có thể định nghĩa \textit{heap cực tiểu} là 
    một danh sách $A$ mà với mọi $i \geq 0$, ta có:
    $$A[i] \leq A[2i+1]$$
    $$A[i] \leq A[2i+2]$$
    Viết lại chương trình sắp xếp vun đống sử dụng \textit{heap cực tiểu}.

    \textbf{Bài 3}. Cho $A$ là một \textit{heap cực đại}, viết chương trình thêm \textbf{một} phần tử mới vào $A$
    sao cho vẫn được một heap cực đại.
    \textit{Ví dụ}. Giả sử $A = [190, 178, 145, 154, 54, 110, 137]$, khi đó:
    \begin{itemize}
        \item \texttt{them\_vao\_heap(A, 12) -> [190, 178, 145, 154, 54, 110, 137, 12]}
        \item \texttt{them\_vao\_heap(A, 160) -> [190, 178, 145, 160, 54, 110, 137, 154]}
    \end{itemize}
\end{document}