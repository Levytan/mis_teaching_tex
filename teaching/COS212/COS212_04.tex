\documentclass[11pt,aspectratio=169]{beamer}
\usepackage{unicode-math}
\usepackage[style=ddmmyyyy]{datetime2}
\usepackage{graphicx}
\usepackage{hyperref}
\usepackage{fontspec}
\usepackage[dvipsnames]{xcolor}
\usepackage{tikz}
\usepackage{soul}
\usepackage{ulem}
\usepackage{ragged2e}
\usepackage{array}

\makeatletter
\def\input@path{{theme/}}
\def\beamer@shrinkfactorinv{1}
\makeatother
\usetheme[
    twocolumntoc,
    allserif,
    palette=sakura
]{Celestia}

%set font
\setmainfont{EB Garamond}[Numbers=OldStyle]
\setsansfont{Noto Sans}[Numbers=OldStyle]
\setmathfont{Asana Math}[Scale=MatchLowercase]
\setmonofont{Cascadia Code}[Scale=MatchLowercase]

\title{Mô hình quan hệ (Relational Model)}
\subtitle{Cơ cở dữ liệu cơ bản \textendash\ COS212}
\institute{Khoa Hệ thống Thông tin Quản lý\\HUTECH}
\author{Lê Thành Văn}
\date{01.12.2025}
%
\AtBeginSection{
  \frame{
    \sectionpage%
  }
}
%
\addtobeamertemplate{block begin}{}{\justifying}%
\setlength{\tabcolsep}{10pt}
\renewcommand{\arraystretch}{1.5}
\begin{document}
    \begin{frame}
        \titlepage%
    \end{frame}
    \begin{frame}
    \tableofcontents[hideallsubsections]%
    \end{frame}
    \section{Cơ sở lý thuyết}
    \subsection{Giới thiệu}
    \begin{frame}{Giới thiệu}
      Mô hình dữ liệu quan hệ (Relational Model) được phát minh bởi E.F.Codd vào năm 1970. 
      Mô hình này cung cấp một cấu trúc dữ liệu đơn giản và đồng bộ đó là khái niệm quan hệ.
    \end{frame}
    \begin{frame}
      Mô hình dựa trên một nền tảng toán học khá vững chắc đó là lý thuyết tập hợp và đại số quan hệ.
    \end{frame}
    \begin{frame}{Các khái niệm}
    Các khái niệm trong mô hình quan hệ:
    \begin{itemize}
      \item Quan hệ (Relation).
      \item Thuộc tính (Attribute).
      \item Bộ (Tuple).
      \item Thể hiện của quan hệ (Instance).
      \item Lược đồ quan hệ (Relation Schema).
    \end{itemize}    
    \end{frame}
    \subsection{Định nghĩa}
    \begin{frame}{Quan hệ \textendash\ Relation}
      \begin{block}{Định nghĩa}
        \textbf{Quan hệ} là một cách tổ chức dữ liệu thành bảng hai chiều.
        Trong đó, mỗi cột được gọi là một thuộc tính (trường), mỗi dòng được gọi là một bộ.
      \end{block}
      ~\\
      Tên của quan hệ được viết in hoa toàn bộ từ và dùng dấu gạch dưới thay cho khoảng trắng.
      \textit{Ví dụ}: SINH\_VIÊN, GIẢNG\_VIÊN, \dots
    \end{frame}
    \begin{frame}{Ví dụ}
      \begin{figure}
        \centering
        \includegraphics[width=0.8\textwidth]{COS212/relation.pdf}
        \caption{\textit{Quan hệ}}\label{fig:relation}
      \end{figure}
    \end{frame}
    \begin{frame}{Thuộc tính \textendash\ Attribute}
      \begin{block}{Định nghĩa}
        \begin{itemize}
          \item Tên các cột của quan hệ.
          \item Mô tả ý nghĩa cho các giá trị tại cột đó.
          \item Tất cả các dữ liệu trong cùng một cột đều có cùng kiểu dữ liệu
        \end{itemize}
      \end{block}
      ~\\
      Tên của thuộc tính được viết thường toàn bộ từ và dùng dấu gạch dưới thay cho khoảng trắng.
      \textit{Ví dụ}: mã\_số\_sinh\_viên, họ\_tên, \dots
    \end{frame}
    \begin{frame}{Thuộc tính \textendash\ Attribute (tiếp)}
      \begin{block}{Tiêu đề}
        \textbf{Tiêu đề} là dòng đầu tiên của một quan hệ, nó chứa tên các thuộc tính của quan hệ đó.
      \end{block}
      \begin{block}{Bậc của quan hệ}
        \textbf{Bậc của quan hệ} là số lượng thuộc tính của một quan hệ.
      \end{block}
      ~\\
      Trong \hyperlink{fig:relation}{ví dụ trên}, quan hệ \textbf{CHI\_NHÁNH} có bậc 4, quan hệ \textbf{NHÂN\_VIÊN} có bậc 7.
    \end{frame}
    \begin{frame}{Miền xác định \textendash\ Domain}
      \begin{block}{Định nghĩa}
        Miền xác định của một thuộc tính là tập hợp tất cả các giá trị mà thuộc tính đó có thể nhận.
        \\[1em]
        Miền xác định của thuộc tính \textit{attr} được ký hiệu là DOM(attr).
      \end{block}
      ~\\
      Trong \hyperlink{fig:relation}{ví dụ trên}, thuộc tính \textbf{giới\_tính} có thể có miền xác định là \textbf{\{Nam, Nữ, Khác\}}.
    \end{frame}
    \begin{frame}{Miền xác định \textendash\ Domain (tiếp)}
      Một thuộc tính (trong một bộ) có thể không xác định được giá trị, ta gọi là \textbf{NULL} (chỗ trống).
    \end{frame}
    \begin{frame}{Bộ giá trị \textendash\ Tuple}
      \begin{block}{Định nghĩa}
        \textbf{Bộ giá trị} là một dòng của quan hệ (trừ tiêu đề) thể hiện dữ liệu cụ thể của các thuộc tính trong quan hệ.
        Các giá trị của bộ được phân cách với nhau bằng dấu phẩy.
        Thứ tự các thành phần trong bộ tương ứng với thứ tự trong dòng tiêu đề.
      \end{block}
    \end{frame}
    \begin{frame}{Bộ giá trị \textendash\ Tuple (tiếp)}
      Một bộ được ký hiệu là $t_v = \langle v_1, v_2, \ldots, v_k\rangle$, trong đó:
      \begin{itemize}
        \item $v_i$ là giá trị của thuộc tính thứ $i$ trong tiêu đề, gọi thuộc tính đó là attr$_i$.
        \item $v_i$ thuộc miền xác định của attr$_i$, hoặc là NULL.
      \end{itemize}
      ~\\
      Ngoài ra, ta còn ký hiệu $v_i = t_v[\text{attr}_i]$.
    \end{frame}
    \begin{frame}{Bộ giá trị \textendash\ Tuple (tiếp)}
      Cho hai bộ của cùng một quan hệ $t_v = \langle v_1, v_2, \ldots, v_k\rangle$ và $t_u = \langle u_1, u_2, \ldots, u_k\rangle$.
      Hai bộ trên được gọi là giống nhau nếu và chỉ nếu:
      \begin{itemize}
        \item $v_i$ và $u_i$ đều khác NULL, với mọi $i$.
        \item $v_i = u_i$, với mọi $i$.
      \end{itemize}
    \end{frame}
    \begin{frame}{Thể hiện của quan hệ \textendash\ Instance}
      \begin{block}{Định nghĩa}
        \textbf{Thể hiện của một quan hệ} là tập hợp các bộ giá trị của quan hệ tại một thời điểm nhất định.
        Ta dùng ký hiệu r(R) để chỉ thể hiện của quan hệ R.
      \end{block}
      \begin{block}{Định nghĩa}
        \textbf{Lực lượng của quan hệ} là số bộ của quan hệ đó. 
      \end{block}
    \end{frame}
    \begin{frame}{Lược đồ quan hệ \textendash\ Relation Schema}
      \begin{block}{Định nghĩa}
        Lược đồ quan hệ được đặc trưng bởi hai thành phần:
        \begin{itemize}
          \item Một tên quan hệ \textbf{R}.
          \item Một danh sách các thuộc tính \textit{attr}$_1$, \textit{attr}$_2$, \dots, \textit{attr}$_n$.
        \end{itemize}
        ~\\
        Ký hiệu: R(attr$_1$, attr$_2$, \dots, attr$_n$)
      \end{block}
    \end{frame}
    \begin{frame}{Tóm tắt ký hiệu}
    \begin{table}
      \begin{tabular}{m{0.6\textwidth}|m{0.3\textwidth}}
        \hline
        \centering \textbf{Mô tả} & \centering \textbf{Ký hiệu} \newlinechar 
        \hline
      \end{tabular}
    \end{table}
    \end{frame}
    \section{Ràng buộc toàn vẹn}
    \begin{frame}{Ràng buộc toàn vẹn}
      \begin{block}{Định nghĩa}
        \textbf{CSDL quan hệ} (Relational Database) là cơ sở dữ liệu sử dụng mô hình quan hệ, 
        nghĩa là, dữ liệu được tổ chức, sắp xếp thành nhiều quan hệ (bảng hai chiều).
      \end{block}
      \begin{block}{Định nghĩa}
        \textbf{Ràng buộc toàn vẹn} là những quy tắc, điều kiện cần được thỏa mãn cho \textbf{mọi thể hiện} của CSDL quan hệ.
      \end{block}
    \end{frame}
    \section{Chuyển đổi từ mô hình thực thể \textendash\ kết hợp}
\end{document}