\documentclass[18pt]{beamer}
\usepackage[style=ddmmyyyy]{datetime2}
\usepackage{fontspec}
\usepackage{unicode-math}
\usepackage{lipsum}
\usepackage{hyperref}

%font setup
%\usefonttheme{serif}
%\setmainfont{Merriweather}
\setsansfont{Fira Sans Light}[Numbers=OldStyle]
\setmathfont{STIX Two Math}[Scale=MatchLowercase]
\setmonofont{Fira Code}%[Scale=MatchLowercase]

\title{There Is No Largest Prime Number}
\date{\Today}
\author{\texorpdfstring{Author\newline\small\url{email@email.com}}{Author}}
\institute{Khoa Hệ thống thông tin quản lý}

\usetheme[block=fill]{mis}

\begin{document}

\begin{frame}
\titlepage
\end{frame}


\begin{frame} 
\frametitle{There Is No Largest Prime Number}
\begin{theorem}
There is no largest prime number. \end{theorem} 
\begin{enumerate} 
\item<1-> Giả sử $p$ là số nguyên tố lớn nhất. 
\item<2-> Gọi $q$ là tích tất cả các số nguyên tố. 
\item<3-> Then $q+1$ is not divisible by any of them. 
\item<4-> But $q + 1$ is greater than $1$, thus divisible by some prime
number not in the first $p$ numbers.
\end{enumerate}
\end{frame}

\begin{frame}{A longer title}
\begin{itemize}
\item one
\item two
\end{itemize}
\end{frame}

\begin{frame}
    \begin{enumerate}
        \item one
        \item two
        \item three 
        \item four
    \end{enumerate}
\end{frame}

\begin{frame}
    Cho biến ngẫu nhiên $\mathcal{X} \sim \mathcal{N}(\mu, \sigma^2)$, khi đó:
    \begin{equation}
        \varphi(x) = \frac{1}{\sigma\sqrt{2\pi}} e^{-\frac{1}{2}\left(\frac{x-\mu}{\sigma}\right)^2}
    \end{equation}
    Nếu $\mu = 0$ và $\sigma = 1$, ta nói $\mathcal{X}$ có phân phối chuẩn chuẩn tắc.
\end{frame}

\begin{frame}[label=conclusion, standout]{Conclusion}
    Awesome slide
\end{frame}

\end{document}