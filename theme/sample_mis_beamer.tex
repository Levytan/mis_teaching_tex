\documentclass{beamer}
\usepackage{fontspec}
\usepackage{unicode-math}
\usepackage{lipsum}

%font setup
\usefonttheme[onlymath]{serif}
\setmainfont{Lora}
%\setsansfont{Fira Sans}
\setmathfont{STIX Two Math}[Scale=MatchLowercase]
\setmonofont{Consolas}[Scale=MatchLowercase]

\title{There Is No Largest Prime Number}
\date[ISPN ’80]{27th International Symposium of Prime Numbers}
\author[Euclid]{Euclid of Alexandria \texttt{euclid@alexandria.edu}}

\usetheme{mis}

\begin{document}

\begin{frame}
\titlepage
\end{frame}


\begin{frame} 
\frametitle{There Is No Largest Prime Number} 
\framesubtitle{The proof uses \textit{reductio ad absurdum}.} 
\begin{theorem}
There is no largest prime number. \end{theorem} 
\begin{enumerate} 
\item<1-> Giả sử $p$ là số nguyên tố lớn nhất. 
\item<2-> Tính $q$ là tích tất cả các số nguyên tố. 
\item<3-> Then $q+1$ is not divisible by any of them. 
\item<4-> But $q + 1$ is greater than $1$, thus divisible by some prime
number not in the first $p$ numbers.
\end{enumerate}
\end{frame}

\begin{frame}{A longer title}
\begin{itemize}
\item one
\item two
\end{itemize}
\end{frame}

\begin{frame}
    \begin{equation}
        \varphi(x) = \int_{\mathbb{R}}e^{-\frac{x^2}{2}} dx
    \end{equation}
\end{frame}

\end{document}